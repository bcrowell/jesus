\documentclass[11pt,twocolumn]{article}
\usepackage{fontspec}
\setmainfont{DejaVu Serif} % This font looks like ass. But it makes greek work.
% Haven't set up Hebrew/Aramaic -- https://tex.stackexchange.com/q/354676/6853

\usepackage{parskip} % https://tex.stackexchange.com/a/57/6853

\usepackage{navigator}
\usepackage{graphicx}
\usepackage{xcolor}

\newcommand{\quotesize}{\normalsize{}}
\newcommand{\maintextquotesize}{\renewcommand{\quotesize}{\large{}}}
\newcommand{\notequotesize}{\renewcommand{\quotesize}{\normalsize{}}}
\newcommand{\comm}[1]{\begingroup \color{black!50} #1\endgroup}
\newenvironment{quotetext}{\begingroup\quotesize}{\endgroup}
\newcommand{\bible}[2]{\begin{quotetext}\textbf{#1} #2\end{quotetext}}
\newcommand{\matthew}[2]{\bible{Matthew #1}{#2}}
\newcommand{\gospelmark}[2]{\bible{Mark #1}{#2}}
\newcommand{\luke}[2]{\bible{Luke #1}{#2}}
\newcommand{\john}[2]{\bible{John #1}{#2}}

\newcommand{\personal}[1]{\emph{Personal opinion:\/ #1}}

\newcommand{\subhead}[1]{\emph{#1}\\}

%----------------------------------------------------------------------------------
% End notes.
%
% This could conceivably conflict with a \danger macro defined in the stix fonts.
% https://tex.stackexchange.com/a/262549/6853
% https://tex.stackexchange.com/a/18160/6853}
% The anchor and jumplink macros are from the navigator package, https://tex.stackexchange.com/a/200073/6853 .
\DeclareFontFamily{U}{stixbbit}{}
\DeclareFontShape{U}{stixbbit}{m}{it}{<-> stix-mathbbit}{}
\DeclareRobustCommand{\stixdangerousbend}{%
  {\usefont{U}{stixbbit}{m}{it}\symbol{"F6}}%
}
\newcommand{\dangerousbend}{\rotatebox[origin=c]{-10}{\stixdangerousbend}}
\newcommand{\link}[2]{\protect\jumplink{anchor-#1}{\textcolor{blue}{#2}}} % navigator package
\newcommand{\note}[1]{\link{note-#1}{\dangerousbend\pageref{note:#1}}\label{notebackref:#1}\anchor{anchor-noteback-#1}}
\newcommand{\notetext}[2]{\textbf{\link{noteback-#1}{\dangerousbend\pageref{notebackref:#1}}\label{note:#1}\anchor{anchor-note-#1}\quad{}#2}\par}
% ... first arg is label, second is title
\newcommand{\notewithoutbackref}[1]{\link{note-#1}{\dangerousbend\pageref{note:#1}}}
% ... use this to refer to a note elsewhere than in the spot where the note's main reference is
\newcommand{\notesummary}[1]{\emph{#1}\par}
\newenvironment{notesection}[1]{
  % #1 is title, such as Notes
  % modeled on activity environment in lmcommon.sty, invoked by begin_notes() in eruby_util.rb
  \setcounter{secnumdepth}{0}          % Don't number this section.
  \section{#1}
  \setcounter{secnumdepth}{2}          % Start numbering sections again.
                                       % The \setcounter{secnumdepth} stuff is the way the author of titlesec suggests doing
                                       % this. Using section* messes up footers & toc. 
  \notequotesize
}%
{
  \maintextquotesize
}
%----------------------------------------------------------------------------------

\begin{document}

The purpose of this document is to construct, for my own study, a life of Jesus with a single narrative
thread, comprised mainly of a backbone of text from contemporary sources such as the gospels and 
Josephus \note{about-this-doc}.

Political timeline:

ca.~72-4 BCE --- Herod the Great is king of Judea, with Rome as his patron. He renovates the Temple.

ca. 4 BCE-39 CE --- Judea is ruled by tetrarchs and an ethnarch, including Herod Antipater (``Antipas'') over Galilee.
      In 34-36 CE, Antipas loses a war with Aretas IV of Nabatea, an Arab trading empire centered on modern Jordan.

41-44 CE --- Herod Agrippa reigns over a reunified Judea. He engages in Roman power politics, sticks up for the Jews,
      and persecutes the early Christians, imprisoning Peter.

66-73 --- First Jewish–Roman War. The temple in Jerusalem is destroyed. Gospel of Mark written around this time.

\begin{section}{John the Baptist}
The Gospel of Mark is likely the first written record of Jesus, and it begins not with a description of Jesus but of John the Baptist.
We know nothing reliable about John's birthplace or date of birth. Luke, in the course of a miraculous birth story, claims that
John came from a priestly family:

\luke{1:5}{There was in the days of Herod, the king of Judea, a certain priest named Zacharias [John's father], of the priestly division of Abijah.}

\comm{
John was a great celebrity of his time and place, meriting a long passage in
Josephus (much longer than the brief mention of Jesus\footnote{Antiquities of the Jews 20-9;
excluding the likely forgery in 18-3}).
Josephus says:\footnote{18-5-2} % https://www.gutenberg.org/files/2848/2848-h/2848-h.htm#link182HCH0005
}

\begin{quotetext}
Now some of the Jews thought that the destruction of Herod [Antipater]'s army [ca.~36 CE, by the Nabateans] came
from God, and that very justly, as a punishment of what he did against
John, that was called the Baptist: for Herod slew him, who was a good
man, and commanded the Jews to exercise virtue, both as to
righteousness towards one another, and piety towards God, and so to
come to baptism; for that the washing [with water] would be acceptable
to him, if they made use of it, not in order to the putting away [or
the remission] of some sins [only], but for the purification of the
body; supposing still that the soul was thoroughly purified beforehand
by righteousness.
\end{quotetext}

\comm{
Josephus's description of the significance of the baptism ritual is probably
distorted and sanitized in order to compromise between John's popularity and Joseph's sympathies
with the Roman-Jewish regime. Mark gives a different description: --
}

\gospelmark{1:4}{John came baptizing in the wilderness and preaching the
baptism of repentance for forgiveness of sins.  All the country of
Judea and all those of Jerusalem went out to him. They were baptized
by him in the Jordan river, confessing their sins.}

\comm{
John's actions and the symbolism of the Jordan were extremely politically provocative to the Judean client regime \note{john-provocative}.
Mark describes John as an ascetic.\note{john-ascetic}
}

\gospelmark{1:6}{John was clothed with camel’s hair and a leather belt around his waist. He ate locusts
and wild honey.}


\comm{
Matthew, unlike Mark, has John prefiguring Jesus's mission by saying
in addition,
}

\matthew{3:2}{``Repent, for the Kingdom of Heaven is at hand!''}

\comm{
and hurling abuse directly in the face of the priestly class as a ``brood of vipers'' (3:7). We don't
know whether John's message was actually apocalyptic or so explicitly antiestablishment.

Luke recounts further social teachings of John (possibly inauthentic \note{john-social}):
}

\luke{3:10}{The multitudes asked him, ``What then must we do?''
 He answered them, ``He who has two coats, let him give to him who has none. He who has food, let him do likewise.''
 Tax collectors also came to be baptized, and they said to him, ``Teacher, what must we do?''
 He said to them, ``Collect no more than that which is appointed to you.''
 Soldiers also asked him, saying, ``What about us? What must we do?''
He said to them, ``Extort from no one by violence, neither accuse anyone wrongfully. Be content with your wages.''}

\comm{
Josephus continues with the story of John's doom:
}

\begin{quotetext}
Now when [many] others came in crowds about him, for
they were very greatly moved [or pleased] by hearing his words, Herod,
who feared lest the great influence John had over the people might put
it into his power and inclination to raise a rebellion, [for they
seemed ready to do any thing he should advise,] thought it best, by
putting him to death, to prevent any mischief he might cause, and not
bring himself into difficulties, by sparing a man who might make him
repent of it when it would be too late. Accordingly he was sent a
prisoner, out of Herod's suspicious temper, to Macherus, the castle I
before mentioned, and was there put to death [around 30 CE]. Now the Jews had an
opinion that the destruction of this army was sent as a punishment
upon Herod, and a mark of God's displeasure to him.
\end{quotetext}

\comm{
John was believed by some to have been the messiah, and he retained a body of followers for
generations after his death.
}


\end{section}

\begin{section}{Jesus's origins and early life}

\comm{
Mark, John, and Paul ignore the question of Jesus's birth.\footnote{Luke and Matthew provide stories of a miraculous birth,
along with a ``massacre of the innocents'' (Matthew 2:16).} Embedded in the miraculous birth stories is the statement
that he is an oldest son, born (ca.~4 BCE) in
very humble circumstances.
}

\luke{2:7}{[Mary] gave birth to her firstborn son. She wrapped him in bands of cloth and laid him in a manger\ldots}

\comm{
The skeptical Nazareans later scornfully recount the background of this uppity local boy (\note{family-interp}):
}

\gospelmark{6:2}{``What is the wisdom that is given to this man, that such
mighty works come about by his hands? Isn’t this the manual laborer, the
son of Mary and brother of James, Joses, Judah, and Simon? Aren’t his
sisters here with us?'' So they were offended at him.}

\comm{
Josephus (20-9-1) confirms that Jesus has a brother James, who was later to become the movement's main leader
in Jerusalem \note{james}:
}

\begin{quotetext} % https://www.gutenberg.org/files/2848/2848-h/2848-h.htm#link202HCH0009
\ldots [the procurator Albinus] assembled the sanhedrim of judges, and brought before them the brother of Jesus, who was called Christ, whose name was James, and some others \ldots [and] delivered them to be stoned
\end{quotetext}

\comm{
Jesus grew up in the tiny village of Nazareth, in Galilee.\note{jesus-galilee}
We don't know whether he learned to read, but he must have had an opportunity
at least to hear the Hebrew Bible and learn it thoroughly \note{literacy}.

 At periodic festivals,
Jews who were able to afford to travel gathered in great crowds at the temple in
Jerusalem:
}

\luke{2:41}{His parents went every year to Jeru\-salem at the feast of the Passover.  When he was twelve years old, they went up to Jerusalem according to the custom of the feast;  and when they had fulfilled the days, as they were returning, the boy Jesus stayed behind in Jerusalem. Joseph and his mother didn’t know it,  but supposing him to be in the company, they went a day’s journey; and they looked for him among their relatives and acquaintances.  When they didn’t find him, they returned to Jerusalem, looking for him.  After three days they found him in the temple, sitting in the middle of the teachers, both listening to them and asking them questions.  All who heard him were amazed at his understanding and his answers.  When they saw him, they were astonished; and his mother said to him, ``Son, why have you treated us this way? Behold, your father and I were anxiously looking for you.''
 He said to them, ``Why were you looking for me? Didn’t you know that I must be in my Father’s house?''}

\comm{
This miracle story may preserve a factual picture of a precocious child getting an education against all odds, or
it may be that Jesus gained his education as a teenager or an adult, like Frederick Douglass. The gospels leap
over Jesus's youth and young adulthood to ca.~28 CE, when Jesus would have been about 31.
}

\luke{3:1}{Now in the fifteenth year of the reign [14-37 CE] of Tiberius
Caesar\ldots the word of God came to John, the son of Zacharias, in
the wilderness.  He came into all the region around the Jordan,
preaching the baptism of repentance for remission of sins.}

\comm{
Among John's followers were a number of people who were later to become the early Christians.
One of the few things we can know securely about the historical Jesus is that he was baptized
by John:
}

\gospelmark{1:9}{In those days, Jesus came from Nazareth of Galilee, and was baptized by John in the Jordan.}

\comm{Others include Simon Peter, apostle and founder of the Church, and his brother:}

\john{1:40}{One of the two [disciples] who heard John [describing 
Jesus as the messiah] and followed him was Andrew, Simon Peter’s
brother. He first found his own brother, Simon, and said to him, “We
have found the Messiah!” (which is, being interpreted, Christ). He
brought him to Jesus. Jesus looked at him and said, “You are Simon the
son of Jonah. You shall be called Cephas” (which is by interpretation,
Peter).}

\comm{Others are described in Acts:}

\bible{Acts 19:1}{While Apollos [an Alexandrian Jew who was a follower of John and then converted in Acts 18] was at Corinth, Paul, having passed through the upper country, came to Ephesus and found certain disciples. He said to them, “Did you receive the Holy Spirit when you believed?”
They said to him, ``No, we haven’t even heard that there is a Holy Spirit.''
 He said, ``Into what then were you baptized?''
They said, ``Into John’s baptism.''
 Paul said, ``John indeed baptized with the baptism of repentance, saying to the people that they should believe in the one who would come after him, that is, in Christ Jesus.''
 When they heard this, they were baptized in the name of the Lord Jesus.}

\comm{
John carves out his own rogue franchise, competing with the temple in Jerusalem by
offering free purification rituals, with a message that is threatening to both Rome and the priesthood.
}

\luke{3:18}{Then with many other exhortations [John] preached good news to the people, but Herod the tetrarch, being reproved by him for Herodias, his brother’s wife, and for all the evil things which Herod had done, added this also to them all, that he shut up John in prison.}

\gospelmark{1:12}{Immediately the Spirit drove [Jesus] out into the
wilderness. He was there in the wild\-erness forty days, tempted by
Satan. He was with the wild animals; and the angels were serving him.\footnote{Luke 4 has a lengthier account of
the temptation, with the devil offering Jesus worldly power. \personal{This is unlikely to have any connection to
Jesus's actual visions and spiritual experiences. It sounds more like the later sanitizing of the early
Christian religion for consumption by Romans, who did not want to be part of a religion that preached the overthrow
of the empire.}}}
\end{section}

\begin{section}{Early mission and calling of the apostles}

\luke{3:23}{Jesus himself, when he began to teach, was about thirty years old\ldots}

\subhead{Preaching in Nazareth}

\gospelmark{1:14}{ Now after John was arrested, Jesus came
into Galilee, preaching the Good News of God’s Kingdom, and saying,
``The time is fulfilled, and God’s Kingdom is at hand! Repent, and
believe in the Good News.''}

\luke{4:14}{Jesus returned in the power of the Spirit into Galilee, and news about him spread through all the surrounding area. \ldots
  He came to Nazareth, where he had been brought up.\footnote{Luke has Jesus reading from a scroll and teaching about Elijah and
Isaiah. \personal{This seems like a later addition by Luke, in an effort to present Jesus as paralleling and fulfilling the prophets. The scene
is absent from Mark.}}}

\luke{4:22}{All testified about him and wondered at the gracious words which proceeded out of his mouth; and they said, ``Isn't this Joseph's son?''}

\luke{4:28}{ They were all filled with wrath in the synagogue as they heard these things.   They rose up, threw him out of the city, and led him to the brow of the hill that their city was built on, that they might throw him off the cliff.   But he, passing through the middle of them, went his way.}

\subhead{Capernaum; The first apostles}

\gospelmark{1:16}{Passing along by the sea of Galilee,\footnote{John 1:44 says Bethsaida.} he saw Simon
  and Andrew, the brother of Simon, casting a net into the sea, for
  they were fishermen.  Jesus said to them, ``Come after me, and I
  will make you into fishers for men.''  Immediately they left their
  nets, and followed him.  Going on a little further from there, he
  saw James the son of Zebedee, and John his brother, who were also in
  the boat mending the nets.  Immediately he called them, and they
  left their father, Zebedee, in the boat with the hired servants, and
  went after him.  They went into Capernaum, and immediately on the
  Sabbath day he entered into the synagogue and taught.  They were
  astonished at his teaching, for he taught them as having authority,
  and not as the scribes \note{jesus-writing}.}

\comm{
Jesus gathers followers and creates his
own illicit knock-off of John's illicit knock-off of the Temple's rites.
Where John baptized people and then sent them home, Jesus gathers followers and 
carries out an itinerant lifestyle ministry. 

The first four apostles are Simon Peter, Andrew, John, and James the Greater. Simon Peter
is traditionally considered the founder of the Catholic Church. Andrew's name is Greek rather than Jewish.
}

\gospelmark{1:23}{Immediately there was in their synagogue [in Capernaum] a man with an unclean spirit, and he cried out,   saying, ``Ha! What do we have to do with you, Jesus, you Nazarene? Have you come to destroy us? I know you who you are: the Holy One of God!''
  Jesus rebuked him, saying, ``Be quiet, and come out of him!''
  The unclean spirit, convulsing him and crying with a loud voice, came out of him.   They were all amazed, so that they questioned among themselves, saying, ``What is this? A new teaching? For with authority he commands even the unclean spirits, and they obey him!''   The report of him went out immediately everywhere into all the region of Galilee and its surrounding area.

  Immediately, when they had come out of the synagogue, they came into the house of Simon and Andrew, with James and John.
   Now Simon's wife's mother\footnote{So Simon Peter is married.} lay sick with a fever, and immediately they told him about her.   He came and took her by the hand and raised her up. The fever left her immediately, and she served them.

  At evening, when the sun had set, they brought to him all who were sick and those who were possessed by demons.   All the city was gathered together at the door.   He healed many who were sick with various diseases and cast out many demons. He didn't allow the demons to speak, because they knew him.}

\gospelmark{1:35}{Early in the morning, while it was still dark, he rose up and went out, and departed into a deserted place, and prayed there.   Simon and those who were with him searched for him.   They found him and told him, ``Everyone is looking for you.''
  He said to them, ``Let's go elsewhere into the next towns, that I may preach there also, because I came out for this reason.''   He went into their synagogues throughout all Galilee, preaching and casting out demons.}

\comm{
It would have been more culturally expected for a healer to set up shop in one place. Jesus doesn't do so, nor does
he accept money. \personal{The reason for the secrecy and hiding is unclear. Following John's execution, Jesus may be
effectively on the run from the law, or at least afraid of attracting attention from the authorities. He may be unwelcome
in Capernaum as a follower of the subversive John, and afraid of being run out of town as he was in Nazareth. Or the secrecy
may be Mark's literary prefiguring of the motif of the messianic secret.}
}

\subhead{The leper and the paralytic; calling Matthew}

\gospelmark{1:40}{A leper came to him, begging him, kneeling down to him, and saying to him, ``If you want to, you can make me clean.''
  Being moved with anger,\footnote{Some sources have ``compassion.''} he stretched out his hand, and touched him, and said to him, ``I want to. Be made clean.''   When he had said this, immediately the leprosy departed from him and he was made clean.   He strictly warned him and immediately sent him out,   and said to him, ``See that you say nothing to anybody, but go show yourself to the priest and offer for your cleansing the things which Moses commanded, for a testimony to them.''
  But he went out, and began to proclaim it much, and to spread about the matter, so that Jesus could no more openly enter into a city, but was outside in desert places. People came to him from everywhere. }

\comm{
The leper is ritually unclean according to the book of Leviticus, and the Temple has a prescribed
and expensive ritual for cleaning him. Jesus's command to go to the Temple is therefore ironic
or an intended slap in the face to the authorities. His initial anger makes sense in terms of this context.
}

\gospelmark{2:1}{When he entered again into Capernaum after some days, it was heard that he was at home.   Immediately many were gathered together, so that there was no more room, not even around the door; and he spoke the word to them.   Four people came, carrying a paralytic to him.   When they could not come near to him for the crowd, they removed the roof where he was. When they had broken it up, they let down the mat that the paralytic was lying on.   Jesus, seeing their faith, said to the paralytic, ``Son, your sins are forgiven you.''

  But there were some of the scribes sitting there and reasoning in their hearts,   ``Why does this man speak blasphemies like that? Who can forgive sins but God alone?''

  Immediately Jesus, perceiving in his spirit that they so reasoned within themselves, said to them, ``Why do you reason these things in your hearts?    Which is easier, to tell the paralytic, `Your sins are forgiven;' or to say, `Arise, and take up your bed, and walk?'    But that you may know that the Son of Man has authority on earth to forgive sins''--he said to the paralytic--    ``I tell you, arise, take up your mat, and go to your house.'' 

He arose, and immediately took up the mat and went out in front of them all, so that they were all amazed and glorified God, saying, ``We never saw anything like this!''}

\gospelmark{2:13}{He went out again by the seaside. All the multitude came to him, and he taught them.   As he passed by, he saw 
[Matthew\footnote{The apostle. Named in Mark as Levi the son of Alphaeus.}] sitting at the tax office. He said to him, ``Follow me.'' And he arose and followed him.}

\subhead{Eating with sinners; parable of the bridegroom; foraging on the sabbath}

\gospelmark{2:15}{  He was reclining at the table in his house, and many tax collectors and sinners sat down with Jesus and his disciples, for there were many, and they followed him.   The scribes and the Pharisees, when they saw that he was eating with the sinners and tax collectors, said to his disciples, ``Why is it that he eats and drinks with tax collectors and sinners?''

  When Jesus heard it, he said to them, ``Those who are healthy have no need for a physician, but those who are sick. I came not to call the righteous, but sinners to repentance.''}

\gospelmark{2:18}{John's disciples and the Pharisees were fasting, and they came and asked him, ``Why do John's disciples and the disciples of the Pharisees fast, but your disciples don't fast?''

  Jesus said to them, ``Can the groomsmen fast while the bridegroom is with them? As long as they have the bridegroom with them, they can't fast.    But the days will come when the bridegroom will be taken away from them, and then they will fast in that day.    No one sews a piece of unshrunk cloth on an old garment, or else the patch shrinks and the new tears away from the old, and a worse hole is made.    No one puts new wine into old wineskins; or else the new wine will burst the skins, and the wine pours out, and the skins will be destroyed; but they put new wine into fresh wineskins.'' }

\gospelmark{2:23}{He was going on the Sabbath day through the grain fields; and his disciples began, as they went, to pluck the ears of grain.   The Pharisees said to him, ``Behold, why do they do that which is not lawful on the Sabbath day?''

  He said to them, ``Did you never read what David did when he had need and was hungry--he, and those who were with him?    How he entered into God's house at the time of Abiathar the high priest, and ate the show bread, which is not lawful to eat except for the priests, and gave also to those who were with him?''

  He said to them, ``The Sabbath was made for man, not man for the Sabbath.    Therefore the Son of Man is lord even of the Sabbath.'' 
}

\comm{
John was an ascetic;
Jesus rejects John's asceticism, but retains an antimaterialist slant.
}

\subhead{The withered hand}

\gospelmark{3:1}{
He entered again into the synagogue, and there was a man there whose hand was withered.   They watched him, whether he would heal him on the Sabbath day, that they might accuse him.   He said to the man whose hand was withered, ``Stand up.''   He said to them, ``Is it lawful on the Sabbath day to do good or to do harm? To save a life or to kill?'' But they were silent.   When he had looked around at them with anger, being grieved at the hardening of their hearts, he said to the man, ``Stretch out your hand.'' He stretched it out, and his hand was restored as healthy as the other.   The Pharisees went out, and immediately conspired with the Herodians against him, how they might destroy him.}

\subhead{The commissioning of the apostles}

\gospelmark{3:7}{
  Jesus withdrew to the sea with his disciples; and a great multitude followed him from Galilee, from Judea,   from Jerusalem, from Idumaea, beyond the Jordan, and those from around Tyre and Sidon. A great multitude, hearing what great things he did, came to him.   He spoke to his disciples that a little boat should stay near him because of the crowd, so that they wouldn't press on him.   For he had healed many, so that as many as had diseases pressed on him that they might touch him.   The unclean spirits, whenever they saw him, fell down before him and cried, ``You are the Son of God!''   He sternly warned them that they should not make him known.
}

\comm{
John the Baptist never deputizes anyone to act on his behalf, but Jesus does.
}

\gospelmark{3:13}{
  He went up into the mountain and called to himself those whom he wanted, and they went to him.   He appointed twelve, that they might be with him, and that he might send them out to preach   and to have authority to heal sicknesses and to cast out demons:   Simon (to whom he gave the name Peter);   James the son of Zebedee; and John, the brother of James, (whom he called Boanerges, which means, Sons of Thunder);   Andrew; Philip; Bartholomew; Matthew; Thomas; James, the son of Alphaeus; Thaddaeus; Simon the Zealot;   and Judas Iscariot, who also betrayed him. 
}

\end{section}

\begin{section}{The sermon on the mount}

\comm{
Matthew 5-7 is the sermon on the mount, the greatest statement of Christian ethics.\note{sermon-on-the-mount} 
Its contents are completely different from the rest of the gospels, which are concerned entirely
with the coming end of the world.\note{new-ethics}
He set a moral example in his itinerant lifestyle ministry, but his conduct is sometimes inconsistent with
the wisdom sayings presented in this sermon.\note{jesus-ethics-conduct}
I've left out some verses argued by Geza Vermes to be inauthentic. % https://www.sermononthemount.org.uk/Background/Authenticity.html
}

\matthew{5:1}{
Seeing the multitudes, he went up onto the mountain. When he had sat down, his disciples came to him. He opened his mouth and taught them, saying, 
}

\subhead{The beatitudes}

\matthew{5:3}{
``Blessed are the poor in spirit,
for theirs is the Kingdom of Heaven.

   Blessed are those who mourn,
for they shall be comforted.

   Blessed are the gentle,
for they shall inherit the earth.

   Blessed are those who hunger and thirst for righteousness,
for they shall be filled.

   Blessed are the merciful,
for they shall obtain mercy.

   Blessed are the pure in heart,
for they shall see God.

   Blessed are the peacemakers,
for they shall be called children of God.

   Blessed are those who have been persecuted for righteousness' sake,
for theirs is the Kingdom of Heaven.\footnote{5:11 about persecution is likely a later addition, since the persecution of Christians
hasn't happened yet.}
}

\matthew{5:12}{
Rejoice, and be exceedingly glad, for great is your reward in heaven. [\ldots]
}

\matthew{5:13}{
You are the salt of the earth, but if the salt has lost its flavor, with what will it be salted? It is then good for nothing, but to be cast out and trodden under the feet of men.
}

\matthew{5:14}{You are the light of the world. A city located on a hill can't be hidden.    Neither do you light a lamp and put it under a measuring basket, but on a stand; and it shines to all who are in the house.    Even so, let your light shine before men, that they may see your good works and glorify your Father who is in heaven.\footnote{5:14-15 is a preview of 9:49.}}

\subhead{Extending the Mosaic law}

\matthew{5:17}{
``Don't think that I came to destroy the law or the prophets. I didn't come to destroy, but to fulfill.    For most certainly, I tell you, until heaven and earth pass away, not even one smallest letter or one tiny pen stroke shall in any way pass away from the law, until all things are accomplished.    Therefore, whoever shall break one of these least commandments and teach others to do so, shall be called least in the Kingdom of Heaven; but whoever shall do and teach them shall be called great in the Kingdom of Heaven.    For I tell you that unless your righteousness exceeds that of the scribes and Pharisees, there is no way you will enter into the Kingdom of Heaven.

   ``You have heard that it was said to the ancient ones, `You shall not murder;' and `Whoever murders will be in danger of the judgment.'    But I tell you that everyone who is angry with his brother without a cause  will be in danger of the judgment. Whoever says to his brother, `Raca!' will be in danger of the council. Whoever says, `You fool!' will be in danger of the fire of Gehenna.

   ``If therefore you are offering your gift at the altar, and there remember that your brother has anything against you,    leave your gift there before the altar, and go your way. First be reconciled to your brother, and then come and offer your gift.    Agree with your adversary quickly while you are with him on the way; lest perhaps the prosecutor deliver you to the judge, and the judge deliver you to the officer, and you be cast into prison.    Most certainly I tell you, you shall by no means get out of there until you have paid the last penny.

   ``You have heard that it was said,  `You shall not commit adultery;'    but I tell you that everyone who gazes at a woman to lust after her has committed adultery with her already in his heart.    If your right eye causes you to stumble, pluck it out and throw it away from you. For it is more profitable for you that one of your members should perish than for your whole body to be cast into Gehenna.    If your right hand causes you to stumble, cut it off, and throw it away from you. For it is more profitable for you that one of your members should perish, than for your whole body to be cast into Gehenna.

   ``It was also said, `Whoever shall put away his wife, let him give her a writing of divorce,'    but I tell you that whoever puts away his wife, except for the cause of sexual immorality, makes her an adulteress; and whoever marries her when she is put away commits adultery.

   ``Again you have heard that it was said to the ancient ones, `You shall not make false vows, but shall perform to the Lord your vows,'    but I tell you, don't swear at all: neither by heaven, for it is the throne of God;    nor by the earth, for it is the footstool of his feet; nor by Jerusalem, for it is the city of the great King.    Neither shall you swear by your head, for you can't make one hair white or black.    But let your `Yes' be `Yes' and your `No' be `No.' Whatever is more than these is of the evil one.

   ``You have heard that it was said, `An eye for an eye, and a tooth for a tooth.'    But I tell you, don't resist him who is evil; but whoever strikes you on your right cheek, turn to him the other also.    If anyone sues you to take away your coat, let him have your cloak also.    Whoever compels you to go one mile, go with him two.    Give to him who asks you, and don't turn away him who desires to borrow from you.

   ``You have heard that it was said, `You shall love your neighbor  and hate your enemy.'    But I tell you, love your enemies, bless those who curse you, do good to those who hate you, and pray for those who mistreat you and persecute you,    that you may be children of your Father who is in heaven. For he makes his sun to rise on the evil and the good, and sends rain on the just and the unjust.    For if you love those who love you, what reward do you have? Don't even the tax collectors do the same?    If you only greet your friends, what more do you do than others? Don't even the tax collectors do the same?    Therefore you shall be perfect, just as your Father in heaven is perfect.
}

\subhead{Prayer}

\matthew{6:1}{
    ``Be careful that you don't do your charitable giving before men, to be seen by them, or else you have no reward from your Father who is in heaven.    Therefore, when you do merciful deeds, don't sound a trumpet before yourself, as the hypocrites do in the synagogues and in the streets, that they may get glory from men. Most certainly I tell you, they have received their reward.    But when you do merciful deeds, don't let your left hand know what your right hand does,    so that your merciful deeds may be in secret, then your Father who sees in secret will reward you openly.
   ``When you pray, you shall not be as the hypocrites, for they love to stand and pray in the synagogues and in the corners of the streets, that they may be seen by men. Most certainly, I tell you, they have received their reward.    But you, when you pray, enter into your inner room, and having shut your door, pray to your Father who is in secret; and your Father who sees in secret will reward you openly.    In praying, don't use vain repetitions as the Gentiles do; for they think that they will be heard for their much speaking.    Therefore don't be like them, for your Father knows what things you need before you ask him.    Pray like this:
}

\matthew{6:9}{
`` `Our Father in heaven, may your name be kept holy.

   Let your Kingdom come.
Let your will be done on earth as it is in heaven.

   Give us today our daily bread.

   Forgive us our debts,
as we also forgive our debtors.

   Bring us not into temptation,
but deliver us from the evil one.

For yours is the Kingdom, the power, and the glory forever. Amen.'
}

\matthew{6:14}{
   ``For if you forgive men their trespasses, your heavenly Father will also forgive you.    But if you don't forgive men their trespasses, neither will your Father forgive your trespasses.

   ``Moreover when you fast, don't be like the hypocrites, with sad faces. For they disfigure their faces that they may be seen by men to be fasting. Most certainly I tell you, they have received their reward.    But you, when you fast, anoint your head and wash your face,    so that you are not seen by men to be fasting, but by your Father who is in secret; and your Father, who sees in secret, will reward you.

\subhead{Antimaterialism}

   ``Don't lay up treasures for yourselves on the earth, where moth and rust consume, and where thieves break through and steal;    but lay up for yourselves treasures in heaven, where neither moth nor rust consume, and where thieves don't break through and steal;    for where your treasure is, there your heart will be also.

   ``The lamp of the body is the eye. If therefore your eye is sound, your whole body will be full of light.    But if your eye is evil, your whole body will be full of darkness. If therefore the light that is in you is darkness, how great is the darkness!

   ``No one can serve two masters, for either he will hate the one and love the other, or else he will be devoted to one and despise the other. You can't serve both God and Mammon.    Therefore I tell you, don't be anxious for your life: what you will eat, or what you will drink; nor yet for your body, what you will wear. Isn't life more than food, and the body more than clothing?    See the birds of the sky, that they don't sow, neither do they reap, nor gather into barns. Your heavenly Father feeds them. Aren't you of much more value than they?

   ``Which of you by being anxious, can add one moment to his lifespan?    Why are you anxious about clothing? Consider the lilies of the field, how they grow. They don't toil, neither do they spin,    yet I tell you that even Solomon in all his glory was not dressed like one of these.    But if God so clothes the grass of the field, which today exists and tomorrow is thrown into the oven, won't he much more clothe you, you of little faith?

   ``Therefore don't be anxious, saying, `What will we eat?', `What will we drink?' or, `With what will we be clothed?'    For the Gentiles seek after all these things; for your heavenly Father knows that you need all these things.    But seek first God's Kingdom and his righteousness; and all these things will be given to you as well.    Therefore don't be anxious for tomorrow, for tomorrow will be anxious for itself. Each day's own evil is sufficient. 
}

\subhead{Don't judge}

\matthew{7:1}{
    ``Don't judge, so that you won't be judged.    For with whatever judgment you judge, you will be judged; and with whatever measure you measure, it will be measured to you.    Why do you see the speck that is in your brother's eye, but don't consider the beam that is in your own eye?    Or how will you tell your brother, `Let me remove the speck from your eye,' and behold, the beam is in your own eye?    You hypocrite! First remove the beam out of your own eye, and then you can see clearly to remove the speck out of your brother's eye.

\subhead{Wisdom sayings}

   ``Don't give that which is holy to the dogs, neither throw your pearls before the pigs, lest perhaps they trample them under their feet, and turn and tear you to pieces.

   ``Ask, and it will be given you. Seek, and you will find. Knock, and it will be opened for you.    For everyone who asks receives. He who seeks finds. To him who knocks it will be opened.    Or who is there among you who, if his son asks him for bread, will give him a stone?    Or if he asks for a fish, who will give him a serpent?    If you then, being evil, know how to give good gifts to your children, how much more will your Father who is in heaven give good things to those who ask him!    Therefore, whatever you desire for men to do to you, you shall also do to them; for this is the law and the prophets.

   ``Enter in by the narrow gate; for the gate is wide and the way is broad that leads to destruction, and there are many who enter in by it.    How narrow is the gate and the way is restricted that leads to life! There are few who find it.

   ``Beware of false prophets, who come to you in sheep's clothing, but inwardly are ravening wolves.    By their fruits you will know them. Do you gather grapes from thorns or figs from thistles?    Even so, every good tree produces good fruit, but the corrupt tree produces evil fruit.    A good tree can't produce evil fruit, neither can a corrupt tree produce good fruit.    Every tree that doesn't grow good fruit is cut down and thrown into the fire.    Therefore by their fruits you will know them.

   ``Not everyone who says to me, `Lord, Lord,' will enter into the Kingdom of Heaven, but he who does the will of my Father who is in heaven.
}

\matthew{7:24}{
  ``Everyone therefore who hears these words of mine and does them, I will liken him to a wise man who built his house on a rock.    The rain came down, the floods came, and the winds blew and beat on that house; and it didn't fall, for it was founded on the rock.    Everyone who hears these words of mine and doesn't do them will be like a foolish man who built his house on the sand.    The rain came down, the floods came, and the winds blew and beat on that house; and it fell--and its fall was great.''
}

\matthew{7:28}{
  When Jesus had finished saying these things, the multitudes were astonished at his teaching,   for he taught them with authority, and not like the scribes.
}

\end{section}


\begin{section}{Parables}
\gospelmark{3:19}{

Then [after the commissioning of the apostles] he came into a house.   The multitude came together again, so that they could not so much as eat bread.   When his friends heard it, they went out to seize him; for they said, ``He is insane.''   The scribes who came down from Jerusalem said, ``He has Beelzebul,'' and, ``By the prince of the demons he casts out the demons.''

  He summoned them and said to them in parables, ``How can Satan cast out Satan?    If a kingdom is divided against itself, that kingdom cannot stand.    If a house is divided against itself, that house cannot stand.    If Satan has risen up against himself, and is divided, he can't stand, but has an end.    But no one can enter into the house of the strong man to plunder unless he first binds the strong man; then he will plunder his house.

   ``Most certainly I tell you, all sins of the descendants of man will be forgiven, including their blasphemies with which they may blaspheme;    but whoever may blaspheme against the Holy Spirit never has forgiveness, but is subject to eternal condemnation.''   --because they said, ``He has an unclean spirit.''

  His mother and his brothers came, and standing outside, they sent to him, calling him.   A multitude was sitting around him, and they told him, ``Behold, your mother, your brothers, and your sisters are outside looking for you.''

  He answered them, ``Who are my mother and my brothers?''   Looking around at those who sat around him, he said, ``Behold, my mother and my brothers!    For whoever does the will of God is my brother, my sister, and mother.''
}

\comm{
Jesus radically attacks the traditional structure of the family.
}

\gospelmark{4:1}{
   Again he began to teach by the seaside. A great multitude was gathered to him, so that he entered into a boat in the sea and sat down. All the multitude were on the land by the sea.   He taught them many things in parables, and told them in his teaching,    ``Listen! Behold, the farmer went out to sow.    As he sowed, some seed fell by the road, and the birds came and devoured it.    Others fell on the rocky ground, where it had little soil, and immediately it sprang up, because it had no depth of soil.    When the sun had risen, it was scorched; and because it had no root, it withered away.    Others fell among the thorns, and the thorns grew up and choked it, and it yielded no fruit.    Others fell into the good ground and yielded fruit, growing up and increasing. Some produced thirty times, some sixty times, and some one hundred times as much.''   He said, ``Whoever has ears to hear, let him hear.''

  When he was alone, those who were around him with the twelve asked him about the parables.   He said to them, ``To you is given the mystery of God's Kingdom, but to those who are outside, all things are done in parables,    that `seeing they may see and not perceive, and hearing they may hear and not understand, lest perhaps they should turn again, and their sins should be forgiven them.' ''

  He said to them, ``Don't you understand this parable? How will you understand all of the parables?    The farmer sows the word.    The ones by the road are the ones where the word is sown; and when they have heard, immediately Satan comes and takes away the word which has been sown in them.    These in the same way are those who are sown on the rocky places, who, when they have heard the word, immediately receive it with joy.    They have no root in themselves, but are short-lived. When oppression or persecution arises because of the word, immediately they stumble.    Others are those who are sown among the thorns. These are those who have heard the word,    and the cares of this age, and the deceitfulness of riches, and the lusts of other things entering in choke the word, and it becomes unfruitful.    Those which were sown on the good ground are those who hear the word, accept it, and bear fruit, some thirty times, some sixty times, and some one hundred times.''

  He said to them, ``Is a lamp brought to be put under a basket  or under a bed? Isn't it put on a stand?    For there is nothing hidden except that it should be made known, neither was anything made secret but that it should come to light.    If any man has ears to hear, let him hear.''

  He said to them, ``Take heed what you hear. With whatever measure you measure, it will be measured to you; and more will be given to you who hear.    For whoever has, to him more will be given; and he who doesn't have, even that which he has will be taken away from him.''

  He said, ``God's Kingdom is as if a man should cast seed on the earth,    and should sleep and rise night and day, and the seed should spring up and grow, though he doesn't know how.    For the earth bears fruit by itself: first the blade, then the ear, then the full grain in the ear.    But when the fruit is ripe, immediately he puts in the sickle, because the harvest has come.''

  He said, ``How will we liken God's Kingdom? Or with what parable will we illustrate it?    It's like a grain of mustard seed, which, when it is sown in the earth, though it is less than all the seeds that are on the earth,    yet when it is sown, grows up and becomes greater than all the herbs, and puts out great branches, so that the birds of the sky can lodge under its shadow.''

  With many such parables he spoke the word to them, as they were able to hear it.   Without a parable he didn't speak to them; but privately to his own disciples he explained everything.

  On that day, when evening had come, he said to them, ``Let's go over to the other side.''   Leaving the multitude, they took him with them, even as he was, in the boat. Other small boats were also with him.   A big wind storm arose, and the waves beat into the boat, so much that the boat was already filled.   He himself was in the stern, asleep on the cushion; and they woke him up and asked him, ``Teacher, don't you care that we are dying?''

  He awoke and rebuked the wind, and said to the sea, ``Peace! Be still!'' The wind ceased and there was a great calm.   He said to them, ``Why are you so afraid? How is it that you have no faith?''

  They were greatly afraid and said to one another, ``Who then is this, that even the wind and the sea obey him?'' 
}

\end{section}

\begin{section}{Miracles}
\subhead{The Gadarene swine}
\gospelmark{5:1}{
   They came to the other side of the sea, into the country of the Gadarenes.   When he had come out of the boat, immediately a man with an unclean spirit met him out of the tombs.   He lived in the tombs. Nobody could bind him any more, not even with chains,   because he had been often bound with fetters and chains, and the chains had been torn apart by him, and the fetters broken in pieces. Nobody had the strength to tame him.   Always, night and day, in the tombs and in the mountains, he was crying out, and cutting himself with stones.   When he saw Jesus from afar, he ran and bowed down to him,   and crying out with a loud voice, he said, ``What have I to do with you, Jesus, you Son of the Most High God? I adjure you by God, don't torment me.''   For he said to him, ``Come out of the man, you unclean spirit!''

  He asked him, ``What is your name?''

He said to him, ``My name is Legion, for we are many.''   He begged him much that he would not send them away out of the country.   Now on the mountainside there was a great herd of pigs feeding.   All the demons begged him, saying, ``Send us into the pigs, that we may enter into them.''

  At once Jesus gave them permission. The unclean spirits came out and entered into the pigs. The herd of about two thousand rushed down the steep bank into the sea, and they were drowned in the sea.   Those who fed the pigs fled, and told it in the city and in the country.
The people came to see what it was that had happened.   They came to Jesus, and saw him who had been possessed by demons sitting, clothed, and in his right mind, even him who had the legion; and they were afraid.   Those who saw it declared to them what happened to him who was possessed by demons, and about the pigs.   They began to beg him to depart from their region.

  As he was entering into the boat, he who had been possessed by demons begged him that he might be with him.   He didn't allow him, but said to him, ``Go to your house, to your friends, and tell them what great things the Lord has done for you and how he had mercy on you.''

  He went his way, and began to proclaim in Decapolis how Jesus had done great things for him, and everyone marveled.
}

\subhead{Jairus' daughter}

\gospelmark{5:21}{
  When Jesus had crossed back over in the boat to the other side, a great multitude was gathered to him; and he was by the sea.   Behold, one of the rulers of the synagogue, Jairus by name, came; and seeing him, he fell at his feet   and begged him much, saying, ``My little daughter is at the point of death. Please come and lay your hands on her, that she may be made healthy, and live.''

  He went with him, and a great multitude followed him, and they pressed upon him on all sides.   A certain woman who had a discharge of blood for twelve years,   and had suffered many things by many physicians, and had spent all that she had, and was no better, but rather grew worse,   having heard the things concerning Jesus, came up behind him in the crowd and touched his clothes.   For she said, ``If I just touch his clothes, I will be made well.''   Immediately the flow of her blood was dried up, and she felt in her body that she was healed of her affliction.

  Immediately Jesus, perceiving in himself that the power had gone out from him, turned around in the crowd and asked, ``Who touched my clothes?''

  His disciples said to him, ``You see the multitude pressing against you, and you say, `Who touched me?' ''

  He looked around to see her who had done this thing.   But the woman, fearing and trembling, knowing what had been done to her, came and fell down before him, and told him all the truth.

  He said to her, ``Daughter, your faith has made you well. Go in peace, and be cured of your disease.''

  While he was still speaking, people came from the synagogue ruler's house, saying, ``Your daughter is dead. Why bother the Teacher any more?''

  But Jesus, when he heard the message spoken, immediately said to the ruler of the synagogue, ``Don't be afraid, only believe.''   He allowed no one to follow him except Peter, James, and John the brother of James.   He came to the synagogue ruler's house, and he saw an uproar, weeping, and great wailing.   When he had entered in, he said to them, ``Why do you make an uproar and weep? The child is not dead, but is asleep.''

  They ridiculed him. But he, having put them all out, took the father of the child, her mother, and those who were with him, and went in where the child was lying.   Taking the child by the hand, he said to her, ``Talitha cumi!'' which means, being interpreted, ``Girl, I tell you, get up!''   Immediately the girl rose up and walked, for she was twelve years old. They were amazed with great amazement.   He strictly ordered them that no one should know this, and commanded that something should be given to her to eat.}

\gospelmark{6:1}{
He went out from there. He came into his own country, and his disciples followed him.   When the Sabbath had come, he began to teach in the synagogue, and many hearing him were astonished, saying, ``Where did this man get these things?'' and, ``What is the wisdom that is given to this man, that such mighty works come about by his hands?   Isn't this the carpenter, the son of Mary and brother of James, Joses, Judah, and Simon? Aren't his sisters here with us?'' So they were offended at him.
  Jesus said to them, ``A prophet is not without honor, except in his own country, and among his own relatives, and in his own house.''   He could do no mighty work there, except that he laid his hands on a few sick people and healed them.   He marveled because of their unbelief.
He went around the villages teaching.   He called to himself the twelve, and began to send them out two by two; and he gave them authority over the unclean spirits.   He commanded them that they should take nothing for their journey, except a staff only: no bread, no wallet, no money in their purse,   but to wear sandals, and not put on two tunics.   He said to them, ``Wherever you enter into a house, stay there until you depart from there.    Whoever will not receive you nor hear you, as you depart from there, shake off the dust that is under your feet for a testimony against them. Assuredly, I tell you, it will be more tolerable for Sodom and Gomorrah in the day of judgment than for that city!'' 

They went out and preached that people should repent.   They cast out many demons, and anointed many with oil who were sick and healed them.
}

Mark 6:14-29 inserts a mythologized account of the killing of John the Baptist.

\subhead{The loaves and the fishes}

\gospelmark{6:30}{
The apostles gathered themselves together to Jesus, and they told him all things, whatever they had done, and whatever they had taught.   He said to them, ``Come away into a deserted place, and rest awhile.'' For there were many coming and going, and they had no leisure so much as to eat.   They went away in the boat to a deserted place by themselves.   They saw them going, and many recognized him and ran there on foot from all the cities. They arrived before them and came together to him.   Jesus came out, saw a great multitude, and he had compassion on them because they were like sheep without a shepherd; and he began to teach them many things.   When it was late in the day, his disciples came to him and said, ``This place is deserted, and it is late in the day.   Send them away, that they may go into the surrounding country and villages and buy themselves bread, for they have nothing to eat.''

  But he answered them, ``You give them something to eat.''
They asked him, ``Shall we go and buy two hundred denarii worth of bread and give them something to eat?''

  He said to them, ``How many loaves do you have? Go see.''
When they knew, they said, ``Five, and two fish.''

  He commanded them that everyone should sit down in groups on the green grass.   They sat down in ranks, by hundreds and by fifties.   He took the five loaves and the two fish; and looking up to heaven, he blessed and broke the loaves, and he gave to his disciples to set before them, and he divided the two fish among them all.   They all ate and were filled.   They took up twelve baskets full of broken pieces and also of the fish.   Those who ate the loaves were five thousand men.

  Immediately he made his disciples get into the boat and go ahead to the other side, to Bethsaida, while he himself sent the multitude away.   After he had taken leave of them, he went up the mountain to pray.

  When evening had come, the boat was in the middle of the sea, and he was alone on the land.   Seeing them distressed in rowing, for the wind was contrary to them, about the fourth watch of the night he came to them, walking on the sea;  and he would have passed by them,   but they, when they saw him walking on the sea, supposed that it was a ghost, and cried out;   for they all saw him and were troubled. But he immediately spoke with them and said to them, ``Cheer up! It is I! Don't be afraid.''   He got into the boat with them; and the wind ceased, and they were very amazed among themselves, and marveled;   for they hadn't understood about the loaves, but their hearts were hardened.

  When they had crossed over, they came to land at Gennesaret and moored to the shore.   When they had come out of the boat, immediately the people recognized him,   and ran around that whole region, and began to bring those who were sick on their mats to where they heard he was.   Wherever he entered--into villages, or into cities, or into the country--they laid the sick in the marketplaces and begged him that they might just touch the fringe of his garment; and as many as touched him were made well.
}

\end{section}

%%%%%%%%%%%%%%%%%%%%%%%%%%%%%%%%%%%%%%%%%%%%%%%%%%%%%%%%%%%%%%%%%%%%%%%%%%%%%%%%%%%%%%%%%%%%%%%%%%%%%%%%%%%%%%%
%%%%%%%%%%%%%%%%%%%%%%%%%%%%%%%%%%%%%%%%%%%%%%%%%%%%%%%%%%%%%%%%%%%%%%%%%%%%%%%%%%%%%%%%%%%%%%%%%%%%%%%%%%%%%%%
%                                        notes
%%%%%%%%%%%%%%%%%%%%%%%%%%%%%%%%%%%%%%%%%%%%%%%%%%%%%%%%%%%%%%%%%%%%%%%%%%%%%%%%%%%%%%%%%%%%%%%%%%%%%%%%%%%%%%%
%%%%%%%%%%%%%%%%%%%%%%%%%%%%%%%%%%%%%%%%%%%%%%%%%%%%%%%%%%%%%%%%%%%%%%%%%%%%%%%%%%%%%%%%%%%%%%%%%%%%%%%%%%%%%%%

\begin{notesection}{Notes}

\notetext{about-this-doc}{About this document}
\notesummary{Criteria for inclusion; sources of translations}
I've added my own notes on points that I was originally unable to understand by reading the gospels,
including some information about the political, social, and historical context. I've attempted to
omit all material that, in my amateur opinion, seems to be a later overlay that would not have
been recognizable to Jesus or his contemporaries. An interesting and well known amateur project along
the same lines is the Thomas Jefferson Bible. Unlike Jefferson, I haven't used scientific plausibility
as a criterion to exclude what otherwise seem to have been real, contemporary psychological experiences.
My criteria are arbitrary and personal, and they have the unfortunate side-effect of cutting out many
wonderful and culturally familiar sayings and events, such as ``Man shall not live by bread alone.''

Unless otherwise noted, all English translations of the Gospels and Acts are from the World English Bible.
I've changed a few phrases to more direct, memorable, or familiar ones, such as ``brood of vipers,''
or ``arrested'' rather than ``taken into custody.''
Translations of Josephus are from Whiston, 1737.

\notetext{john-provocative}{John's provocation}
\notesummary{John's actions and the symbolism of the Jordan were extremely politically provocative to the Judean client regime.}
The client regime claimed a monopoly on purification rituals,
which included both ablutions by the priests and lucrative blood sacrifices.
John was not only challenging this monopoly but also (as Josephus carefully omits), by doing them in the
Jordan, evoking a politically supercharged echo of the return of the Jews to the promised land:

\bible{Numbers 27:12}{Yahweh said to Moses, ``Go up into this mountain of Abarim, and see the land which I have given to the children of Israel.''}

There follows in Numbers 27-29 an extremely lengthy list of commands about animal sacrifices to be made to God
once the people have been returned to the promised land. In Deuteronomy 34, God gives Moses another view of the
promised land, reiterating that it is for the offspring of Abraham --- i.e., in the ears of John's followers, not
to the Romans.
Moses's successor Joshua then miraculously crosses the Jordan (Joshua 6) and reconquers Canaan from the evil foreigners.

\notetext{john-ascetic}{John's asceticism}
A strain of asceticism was one of many currents of thought that were in the air among the many
contending schools of Judaism that included the Essene sect. It has been suggested that John was an Essene, but we don't know.

\notetext{john-social}{Authenticity of John's social teachings in Matthew}

\personal{The social teachings in Luke 3:10 seem unlikely to be authentic. John preached to Jews in the wilderness along the Jordan river. What were
tax collectors and Roman soldiers doing there, among that group? Unlike Jesus, John did not, as suggested here,
gather people around him to live his teachings by sharing the necessities of daily life.}

\notetext{family-interp}{Family background}

The Nazareans in Mark 6:2 are insulting Jesus by referring to him as Mary's son. As the eldest brother, he should be referred to as Joseph's son.
The implication is that he is illegitimate.\footnote{Aslan, p.~37}  See also Matthew 13:55-56 and Acts 1:14. 
The word τέκτονος, usually translated as ``carpenter,'' can also just mean a landless manual laborer\footnote{Crossan}
or be Roman slang for an ignorant peasant.\footnote{Aslan, p. 34} 

\notetext{james}{Jesus's brother James the Just}

After the crucifixion, Jesus's brother James took over his mission in Jerusalem.
He managed to coexist for a long time with the Temple while beginning to build
what would become the Christian church. He engaged in a bitter and long-running epistolary
and in-person conflict with Paul over the direction of the movement, and especially about the extent to
which the Mosaic law had to be followed. This resulted in the Council of Jerusalem in 50 CE.

\notetext{jesus-galilee}{Galilee}
As a Galilean, Jesus would have spoken Aramaic
with a distinctive accent.  The Galileans had a reputation as
flinty hill people who didn't pay their taxes and were
anticlerical and resentful of Judea and the Temple.  Galilee produced
many bandits (Greek singular λῃστής), who in some cases may have had
political or Robin Hood overtones.

\notetext{literacy}{Jesus' ability to read}
Any opportunities for Jesus to learn to read or write would have been
extremely scarce, hence the disbelief expressed in Mark 6:2. Most
experts believe that Jesus was unable to read.
Jesus later shows a deep knowledge of the Hebrew Bible and is able to debate its
fine points skillfully. He would probably have had to gain this knowledge by listening in
a synagogue, possibly in Nazareth, if the tiny town had one.\footnote{Aslan, p.~35, claims that no such synagogue existed. The
synagogue at Capernaum, which the gospels record Jesus as visiting as an adult, was 40 km away.}

\notetext{jesus-writing}{Scribes; Jesus' ability to write}

If Jesus could write, it was
probably at the level of ``craftsman's literacy,'' such as the ability
to record business records. John, in the course of the story of the woman
taken in adultery, says,

\john{8:6} But Jesus stooped down and wrote on the ground with his finger.

The verb translated here as ``wrote'' is κατέγραφεν, which could mean either that he wrote or that he drew.

Writing at the highest level of literacy
was more like a specialized profession, that of a scribe. Because
scribes were often officials in the hated Roman regime, analogous to lawyers and bureaucrats, most
references to them in the NT are negative, as in Mark 1:16. A scribe can also indicate one
who is literate and religiously learned, and this class of people was also
suspect because they were part of a leisure class associated with the exploitative practices of the Temple.

Jesus would have had no reason to write down any of his teachings, either himself or with the
help of a scribe. His followers were illiterate, and the Kingdom of God was coming, so there was
no need to record anything for future generations.

\notetext{sermon-on-the-mount}{The sermon on the mount}

The sermon on the mount in Matthew 5-7 summarizes nearly everything that Jesus has to say about ethics.
(There is also a shorter sermon on the plain, Luke 6:20-49.) It reads as a greatest-hits compilation of his most pithy and
memorable wisdom sayings. There is a hypothesis that these sayings in Matthew and Luke come from a common source called Q.

\notetext{new-ethics}{Jesus's new ethics}
\notesummary{Among the synoptic gospels, almost all of Jesus's novel ethical teachings are contained only in the sermon on the mount
(or the shorter sermon on the plain), and not in the (likely earlier) gospel of Mark.}

If a little green man from outer space were to read the gospels, not having any previous cultural indoctrination, he would 
say that Jesus's focus was solely on preparing for the end of the world, and had almost nothing to say about ethics beyond
stern reminders to follow the preexisting Mosaic Law. If Mark and John were specific individuals, then evidently their entire picture
of the new religion contained almost nothing worth recording in terms of new moral teachings.

Don't judge other people: Matthew 7:1. (The story of the woman taken in adultery in John 8 contains
a similar message, but most scholars think it was a later addition.)

No divorce: Matthew 5:31. Its duplication (with somewhat different legalistic conditions) in Mark 10:10 is the
only clear exception to the rule.

Legalistic: Don't get involved in lawsuits, and don't swear oaths.

Emotional: Peace; forgiveness; turn the other cheek; love your enemies; anger is sinful. Lust is sinful.

Antimaterialism: This is not really novel but more a continuation of John the Baptist's teachings: a less extreme version
of John's asceticism, along with a belief that
material gains were useless since the end of the world was coming. Mark 10:21-25 have a similar antimaterialist
message, but seem like a clear later interpolation rather than anything Jesus would have said, since they include
the admonition to ``follow me, taking up the cross.''

\notetext{jesus-ethics-conduct}{Jesus's ethical example contrasted with the sermon on the mount}
\notesummary{He set a moral example in his itinerant lifestyle ministry, but his conduct is sometimes inconsistent with
the wisdom sayings presented in this sermon.}

Jesus is sometimes violent (Mark 11:15, the assault on the moneychangers in the temple),
angry (Mark 1:41, anger with the leper), and in his religious displays sometimes ostentatious (Mark 11:9, the entrance
into Jerusalem).

\end{notesection}

\end{document}
